\subsection{Model Diagnostics}
By using a linear regression model for calculating the loss and gain $\beta $ 
values in the neural (BOLD) signals, we have made the following underlying 
assumptions:
\begin{enumerate}
    \item Normailty of errors
    \item Homoscedasticity (constant variance) or errors  
    \item Linearity of relationship between the explanatory and response
    \item Statistical independence of errors   
    \item All voxels and all subjects follow the same linear model. 
\end{enumerate}
The first four assumption can be motivated by using the residuals as a proxy
for the errors since the errors are unobservable. Here we check the first 
three assumptions by following the standard regression diagnostics plots and 
analysis. The fourth condition, checking for independence of errors, is 
equivalent to checking that no correlation exists between consecutive errors 
in our time serires structure. A simple explanation is given in 
\textit{Simplifications of Model}, so it is leaved out here. For the fifth 
assumption, it may that there are some observations/conditions that do not 
obey the linear model. The methods used to detect these usual observations is 
further discussed in \textit{Outlier Detection}, so it will be left out here. 

\subsubsection{Linearity and Constant Variance}
The residual versus fitted values plot \ref{fig:residual_vs_fitted} is a 
standard plot for observing constant variance of errors and potential deviations
from linearity

\begin{figure}[ht]
\centering
\includegraphics[scale=0.5]{figures/res_fitted}  
\caption{Residual vs Fitted Values (mean across time for subject 1)}
\label{fig:residual_vs_fitted}
\end{figure}

From the figure we can see the residual see to be centered around zero; however.
the variance of the residuals seems to increase as the fitted values increase. 
This perhaps suggests a variance-stabilizing transformation to the response 
vector.  

\subsubsection{Normality}
Normality of the errors is checked by checking the normality of the residuals. 
The normal qq-plot (Quantile-Quantile Plot) \ref{fig:qqplot} shows quantiles
of the sorted residuals averaged across time against the normal theoretical 
quantiles. Ideally, the relationship should be linear. Here we see that there
appears to be heavy tails with an abnormal amount of residuals being zero. In
practice, this makes sense because a large part of the regression produces beta
values that are zero: voxels may not be mapping to the empty regions. While 
the residuals are non-normal and requires attention, the linear-squares 
estimator is still a good approximation in that it is the best linear unbiased 
estimator.

\begin{figure}[ht]
\centering
\includegraphics[scale=0.5]{figures/qqplot}  
\caption{Normal QQ plot of the average residuals across time.}
\label{fig:qqplot}
\end{figure}



